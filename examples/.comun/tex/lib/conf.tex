\usepackage[T1]{fontenc}

% Geometría de la página
\usepackage[
    left=2.5cm,
    right=2.5cm,
    top=2.0cm,
    bottom=2.0cm
]{geometry}

% Interlineado

	\usepackage{setspace}

	\renewcommand{\baselinestretch}{1}

% Paquetes matemáticos
\usepackage{amsmath, amssymb}
\usepackage{amsthm}

	\newtheorem{theorem}{Teorema}
	\newtheorem{proposition}[theorem]{Proposición}
	\newtheorem{pregunta}[theorem]{Pregunta}
	\newtheorem{definition}[theorem]{Definición}
	\newtheorem{lemma}[theorem]{Lema}
	\newtheorem{example}[theorem]{Ejemplo}
	\newtheorem{observacion}[theorem]{Observación}
	\newtheorem{notacion}[theorem]{Notación}

\usepackage{enumitem}

\usepackage[spanish]{babel}

% TikZ
\usepackage{tikz}
\usetikzlibrary{cd}
\usetikzlibrary{babel}

% UTF-8
\usepackage[utf8]{inputenc}

% Paquetes de fecha y hora
\usepackage{datetime2}

% Texto justificado
\usepackage{flushend}

% Tamaño de letra personalizado
\usepackage{anyfontsize}

%%floor and ceilling
\usepackage{mathtools}

%% includegraphics command.
\usepackage{graphicx}

% Extended theorem environments
% \usepackage{amsthm}

% Tamaño de sección
\usepackage{sectsty}
    \sectionfont{\fontsize{13}{11}\selectfont}

% Floats
\usepackage{float}

% Fuentes matemáticas adicionales
\usepackage{mathrsfs}

% Tabla de contenidos
\usepackage[nottoc]{tocbibind}

% Numeración de figuras
\usepackage{chngcntr}
    \counterwithin{figure}{section}

% Configuración de encabezado y pie de página
\usepackage{fancyhdr}
\newcounter{tarea}
\newcounter{lista}
\pagestyle{fancy}
    \fancyhf{}
	\rhead{\titulo}
    %\rhead{\fecha}
    \lhead{\autor}
    \rfoot{\textit{\textbf{\thepage}}}

% Tamaño de indentación
\setlength\parindent{1cm}

% Hipervínculos
\usepackage{hyperref}

% Referencias inteligentes
\usepackage{cleveref}

